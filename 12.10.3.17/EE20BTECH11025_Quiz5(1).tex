\documentclass[journal,12pt,twocolumn]{IEEEtran}
\usepackage{setspace}
\usepackage{gensymb}
\singlespacing
\usepackage[cmex10]{amsmath}
\usepackage{amsthm}
\usepackage{mathrsfs}
\usepackage{txfonts}
\usepackage{stfloats}
\usepackage{bm}
\usepackage{cite}
\usepackage{cases}
\usepackage{subfig}
\usepackage{longtable}
\usepackage{multirow}
\usepackage{enumitem}
\usepackage{mathtools}
\usepackage{steinmetz}
\usepackage{tikz}
\usepackage{circuitikz}
\usepackage{verbatim}
\usepackage{tfrupee}
\usepackage[breaklinks=true]{hyperref}
\usepackage{tkz-euclide}
\usetikzlibrary{calc,math}
\usepackage{listings}
    \usepackage{color}                                            %%
    \usepackage{array}                                            %%
    \usepackage{longtable}                                        %%
    \usepackage{calc}                                             %%
    \usepackage{multirow}                                         %%
    \usepackage{hhline}                                           %%
    \usepackage{ifthen}                                           %%
  %optionally (for landscape tables embedded in another document): %%
    \usepackage{lscape}     
\usepackage{multicol}
\usepackage{chngcntr}
\DeclareMathOperator*{\Res}{Res}
\renewcommand\thesection{\arabic{section}}
\renewcommand\thesubsection{\thesection.\arabic{subsection}}
\renewcommand\thesubsubsection{\thesubsection.\arabic{subsubsection}}

\renewcommand\thesectiondis{\arabic{section}}
\renewcommand\thesubsectiondis{\thesectiondis.\arabic{subsection}}
\renewcommand\thesubsubsectiondis{\thesubsectiondis.\arabic{subsubsection}}

% correct bad hyphenation here
\hyphenation{op-tical net-works semi-conduc-tor}
\def\inputGnumericTable{}                                 %%

\lstset{
frame=single, 
breaklines=true,
columns=fullflexible
}

\begin{document}


\newtheorem{theorem}{Theorem}[section]
\newtheorem{problem}{Problem}
\newtheorem{proposition}{Proposition}[section]
\newtheorem{lemma}{Lemma}[section]
\newtheorem{corollary}[theorem]{Corollary}
\newtheorem{example}{Example}[section]
\newtheorem{definition}[problem]{Definition}
\newcommand{\BEQA}{\begin{eqnarray}}
\newcommand{\EEQA}{\end{eqnarray}}
\newcommand{\define}{\stackrel{\triangle}{=}}

\bibliographystyle{IEEEtran}
\providecommand{\mbf}{\mathbf}
\providecommand{\pr}[1]{\ensuremath{\Pr\left(#1\right)}}
\providecommand{\qfunc}[1]{\ensuremath{Q\left(#1\right)}}
\providecommand{\sbrak}[1]{\ensuremath{{}\left[#1\right]}}
\providecommand{\lsbrak}[1]{\ensuremath{{}\left[#1\right.}}
\providecommand{\rsbrak}[1]{\ensuremath{{}\left.#1\right]}}
\providecommand{\brak}[1]{\ensuremath{\left(#1\right)}}
\providecommand{\lbrak}[1]{\ensuremath{\left(#1\right.}}
\providecommand{\rbrak}[1]{\ensuremath{\left.#1\right)}}
\providecommand{\cbrak}[1]{\ensuremath{\left\{#1\right\}}}
\providecommand{\lcbrak}[1]{\ensuremath{\left\{#1\right.}}
\providecommand{\rcbrak}[1]{\ensuremath{\left.#1\right\}}}
\theoremstyle{remark}
\newtheorem{rem}{Remark}
\newcommand{\sgn}{\mathop{\mathrm{sgn}}}
\providecommand{\abs}[1]{\left\vert#1\right\vert}
\providecommand{\res}[1]{\Res\displaylimits_{#1}} 
\providecommand{\norm}[1]{\left\lVert#1\right\rVert}
\providecommand{\mtx}[1]{\mathbf{#1}}
\providecommand{\mean}[1]{E\left[ #1 \right]}
\providecommand{\fourier}{\overset{\mathcal{F}}{ \rightleftharpoons}}
\providecommand{\system}{\overset{\mathcal{H}}{ \longleftrightarrow}}
\newcommand{\solution}{\noindent \textbf{Solution: }}
\newcommand{\cosec}{\,\text{cosec}\,}
\providecommand{\dec}[2]{\ensuremath{\overset{#1}{\underset{#2}{\gtrless}}}}
\newcommand{\myvec}[1]{\ensuremath{\begin{pmatrix}#1\end{pmatrix}}}
\newcommand{\mydet}[1]{\ensuremath{\begin{vmatrix}#1\end{vmatrix}}}
\numberwithin{equation}{subsection}
\makeatletter
\@addtoreset{figure}{problem}
\makeatother

\let\StandardTheFigure\thefigure
\let\vec\mathbf
\renewcommand{\thefigure}{\theproblem}



\def\putbox#1#2#3{\makebox[0in][l]{\makebox[#1][l]{}\raisebox{\baselineskip}[0in][0in]{\raisebox{#2}[0in][0in]{#3}}}}
     \def\rightbox#1{\makebox[0in][r]{#1}}
     \def\centbox#1{\makebox[0in]{#1}}
     \def\topbox#1{\raisebox{-\baselineskip}[0in][0in]{#1}}
     \def\midbox#1{\raisebox{-0.5\baselineskip}[0in][0in]{#1}}

\vspace{3cm}


\title{Assignment 1}
\author{Jaswanth Chowdary Madala}





% make the title area
\maketitle

\newpage

%\tableofcontents

\bigskip

\renewcommand{\thefigure}{\theenumi}
\renewcommand{\thetable}{\theenumi}


\begin{enumerate}

\item Show that the vectors $2\hat{i}-\hat{j}+\hat{k}$, $\hat{i}-3\hat{j}-5\hat{k}$ and $3\hat{i}-4\hat{j}-4\hat{k}$ form the vertices of a right angled triangle.

\textbf{Solution:} Let us first check whether the given points form a triangle. Let us consider, 
\begin{align}
\vec{A}\myvec{2\\-1\\1} \, \vec{B}\myvec{1\\-3\\-5} \, \vec{C}\myvec{3\\-4\\-4} 
\end{align}
To check whether the points $\vec{A},\vec{B},\vec{C}$ form a triangle, we find the rank of the matrix $\myvec{\vec{A}&\vec{B}&\vec{C}}$

\begin{align}
\myvec{2&1&3\\-1&-3&-4\\1&-5&-4}
\end{align}
$R_{2} \longrightarrow R_{2}+\dfrac{1}{2}R_{1}, \, R_{3} \longrightarrow R_{3}-\dfrac{1}{2}R_{1}$
\begin{align}
\myvec{2&1&3\\ \\0&-\dfrac{5}{2}&-\dfrac{5}{2}\\\\0&-\dfrac{11}{2}&-\dfrac{11}{2}}
\end{align}
For a right angled triangle $ABC$ which is right angled at $\vec{A}$  
\begin{align}
\brak{\vec{B}-\vec{A}}^{\top}\brak{\vec{C}-\vec{A}} &= 0
\end{align}
$R_{3} \longrightarrow R_{3}-\dfrac{11}{5}R_{2}$
\begin{align}
\myvec{2&1&3\\ \\0&-\dfrac{5}{2}&-\dfrac{5}{2}\\\\0&0&0}
\end{align}
The rank of the matrix is 2 and the points are in 3-Dimensional space, So the points $\vec{A},\vec{B},\vec{C}$ form a triangle.\\

Now, we check whether the triangle is right angled at any of the vertices -
$ \vec{A}, \vec{B}, \vec{C}$
 
\begin{enumerate}
\item checking whether the triangle is right angled at $\vec{A}$
\begin{align}
\vec{B}-\vec{A} &= \myvec{-1\\-2\\-6} \\
\vec{C}-\vec{A} &= \myvec{1\\-3\\-5} \\
\brak{\vec{B}-\vec{A}}^{\top}\brak{\vec{C}-\vec{A}} &= \myvec{-1&-2&-6}\myvec{1\\-3\\-5} = 35\\
\brak{\vec{B}-\vec{A}}^{\top}\brak{\vec{C}-\vec{A}} &\neq 0
\end{align}
The triangle is not right angled at $\vec{A}$.\\

\item checking whether the triangle is right angled at $\vec{B}$
\begin{align}
\vec{A}-\vec{B} &= \myvec{1\\2\\6} \\
\vec{C}-\vec{B} &= \myvec{2\\-1\\1} 
\end{align}
\begin{align}
\brak{\vec{A}-\vec{B}}^{\top}\brak{\vec{C}-\vec{B}} &= \myvec{1&2&6}\myvec{2\\-1\\1} = 6\\
\brak{\vec{A}-\vec{B}}^{\top}\brak{\vec{C}-\vec{B}} &\neq 0
\end{align}
The triangle is not right angled at $\vec{B}$.\\

\item checking whether the triangle is right angled at $\vec{C}$
\begin{align}
\vec{A}-\vec{C} &= \myvec{-1\\3\\5} \\
\vec{B}-\vec{C} &= \myvec{-2\\1\\-1} \\
\brak{\vec{A}-\vec{C}}^{\top}\brak{\vec{B}-\vec{C}} &= \myvec{-1&3&5}\myvec{-2\\1\\-1} = 0\\
\brak{\vec{A}-\vec{C}}^{\top}\brak{\vec{B}-\vec{C}} &= 0
\end{align}
Hence the triangle is right angled at $\vec{C}$.
\end{enumerate}
\end{enumerate}

\end{document}



