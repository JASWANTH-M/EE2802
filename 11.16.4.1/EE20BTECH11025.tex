\documentclass[journal,12pt,twocolumn]{IEEEtran}
\usepackage{romannum}
\usepackage{float}
\usepackage{setspace}
\usepackage{gensymb}
\singlespacing
\usepackage[cmex10]{amsmath}
\usepackage{amsthm}
\usepackage{mathrsfs}
\usepackage{txfonts}
\usepackage{stfloats}
\usepackage{bm}
\usepackage{cite}
\usepackage{cases}
\usepackage{subfig}
\usepackage{longtable}
\usepackage{multirow}
\usepackage{enumitem}
\usepackage{mathtools}
\usepackage{steinmetz}
\usepackage{tikz}
\usepackage{circuitikz}
\usepackage{verbatim}
\usepackage{tfrupee}
\usepackage[breaklinks=true]{hyperref}
\usepackage{tkz-euclide}
\usetikzlibrary{calc,math}
\usepackage{listings}
    \usepackage{color}                                            %%
    \usepackage{array}                                            %%
    \usepackage{longtable}                                        %%
    \usepackage{calc}                                             %%
    \usepackage{multirow}                                         %%
    \usepackage{hhline}                                           %%
    \usepackage{ifthen}                                           %%
  %optionally (for landscape tables embedded in another document): %%
    \usepackage{lscape}     
\usepackage{multicol}
\usepackage{chngcntr}
\DeclareMathOperator*{\Res}{Res}
\renewcommand\thesection{\arabic{section}}
\renewcommand\thesubsection{\thesection.\arabic{subsection}}
\renewcommand\thesubsubsection{\thesubsection.\arabic{subsubsection}}

\renewcommand\thesectiondis{\arabic{section}}
\renewcommand\thesubsectiondis{\thesectiondis.\arabic{subsection}}
\renewcommand\thesubsubsectiondis{\thesubsectiondis.\arabic{subsubsection}}

% correct bad hyphenation here
\hyphenation{op-tical net-works semi-conduc-tor}
\def\inputGnumericTable{}                                 %%

\lstset{
frame=single, 
breaklines=true,
columns=fullflexible
}

\begin{document}


\newtheorem{theorem}{Theorem}[section]
\newtheorem{problem}{Problem}
\newtheorem{proposition}{Proposition}[section]
\newtheorem{lemma}{Lemma}[section]
\newtheorem{corollary}[theorem]{Corollary}
\newtheorem{example}{Example}[section]
\newtheorem{definition}[problem]{Definition}
\newcommand{\BEQA}{\begin{eqnarray}}
\newcommand{\EEQA}{\end{eqnarray}}
\newcommand{\define}{\stackrel{\triangle}{=}}

\bibliographystyle{IEEEtran}
\providecommand{\mbf}{\mathbf}
\providecommand{\pr}[1]{\ensuremath{\Pr\left(#1\right)}}
\providecommand{\qfunc}[1]{\ensuremath{Q\left(#1\right)}}
\providecommand{\sbrak}[1]{\ensuremath{{}\left[#1\right]}}
\providecommand{\lsbrak}[1]{\ensuremath{{}\left[#1\right.}}
\providecommand{\rsbrak}[1]{\ensuremath{{}\left.#1\right]}}
\providecommand{\brak}[1]{\ensuremath{\left(#1\right)}}
\providecommand{\lbrak}[1]{\ensuremath{\left(#1\right.}}
\providecommand{\rbrak}[1]{\ensuremath{\left.#1\right)}}
\providecommand{\cbrak}[1]{\ensuremath{\left\{#1\right\}}}
\providecommand{\lcbrak}[1]{\ensuremath{\left\{#1\right.}}
\providecommand{\rcbrak}[1]{\ensuremath{\left.#1\right\}}}
\theoremstyle{remark}
\newtheorem{rem}{Remark}
\newcommand{\sgn}{\mathop{\mathrm{sgn}}}
\providecommand{\abs}[1]{\left\vert#1\right\vert}
\providecommand{\res}[1]{\Res\displaylimits_{#1}} 
\providecommand{\norm}[1]{\left\lVert#1\right\rVert}
\providecommand{\mtx}[1]{\mathbf{#1}}
\providecommand{\mean}[1]{E\left[ #1 \right]}
\providecommand{\fourier}{\overset{\mathcal{F}}{ \rightleftharpoons}}
\providecommand{\system}{\overset{\mathcal{H}}{ \longleftrightarrow}}
\newcommand{\solution}{\noindent \textbf{Solution: }}
\newcommand{\cosec}{\,\text{cosec}\,}
\providecommand{\dec}[2]{\ensuremath{\overset{#1}{\underset{#2}{\gtrless}}}}
\newcommand{\myvec}[1]{\ensuremath{\begin{pmatrix}#1\end{pmatrix}}}
\newcommand{\mydet}[1]{\ensuremath{\begin{vmatrix}#1\end{vmatrix}}}
\numberwithin{equation}{subsection}
\makeatletter
\@addtoreset{figure}{problem}
\makeatother

\let\StandardTheFigure\thefigure
\let\vec\mathbf
\renewcommand{\thefigure}{\theproblem}



\def\putbox#1#2#3{\makebox[0in][l]{\makebox[#1][l]{}\raisebox{\baselineskip}[0in][0in]{\raisebox{#2}[0in][0in]{#3}}}}
     \def\rightbox#1{\makebox[0in][r]{#1}}
     \def\centbox#1{\makebox[0in]{#1}}
     \def\topbox#1{\raisebox{-\baselineskip}[0in][0in]{#1}}
     \def\midbox#1{\raisebox{-0.5\baselineskip}[0in][0in]{#1}}

\vspace{3cm}


\title{Assignment 1}
\author{Jaswanth Chowdary Madala}





% make the title area
\maketitle

\newpage

%\tableofcontents

\bigskip

\renewcommand{\thefigure}{\theenumi}
\renewcommand{\thetable}{\theenumi}



\begin{enumerate}
\item A box contains 10 red marbles, 20 blue marbles and 30 green marbles. 5 marbles
are drawn from the box, what is the probability that
\begin{enumerate}
\item all will be blue?
\item atleast one will be green?
\end{enumerate}
\textbf{Solution:}

\textbf{Lemma:}
Consider the generalized problem, where there are total $N$ marbles - $P$ red, $Q$ blue, $R$ green. We need to find the probability of the event where $n$ marbles drawn such that there are - $p$ red, $q$ blue, $r$ green marbles.\\

Consider the random variable $X_i$, denoting the ith draw, $i \in \{1,2,\cdots, n\}$ as shown in the table \ref{tab:1}
\begin{table}[h]
\centering
%%%%%%%%%%%%%%%%%%%%%%%%%%%%%%%%%%%%%%%%%%%%%%%%%%%%%%%%%%%%%%%%%%%%%%
%%                                                                  %%
%%  This is a LaTeX2e table fragment exported from Gnumeric.        %%
%%                                                                  %%
%%%%%%%%%%%%%%%%%%%%%%%%%%%%%%%%%%%%%%%%%%%%%%%%%%%%%%%%%%%%%%%%%%%%%%

\begin{center}
\begin{tabular}{|c|c|c|}
\hline
\textbf{RV} & \textbf{Description} & \textbf{Probabilities}\\ \hline
$X = 0$	    &   Monday & $\frac{1}{7}$\\ \hline
$X = 1$	    &   Tuesday & $\frac{1}{7}$\\ \hline
$X = 2$	    &   Wednesday & $\frac{1}{7}$\\ \hline
$X = 3$	    &   Thursday & $\frac{1}{7}$\\ \hline
$X = 4$	    &   Friday & $\frac{1}{7}$\\ \hline
$X = 5$	    &   Saturday & $\frac{1}{7}$\\ \hline
$X = 6$	    &   Sunday & $\frac{1}{7}$\\ \hline
\end{tabular}
\end{center}

\caption{Random variables $X_i$}
\label{tab:1}
\end{table}

The probability that the first $p$ draws are red, next $q$ draws are blue, next $r$ draws are green is given by the expression,
\begin{align}
\pr{X_1 = 0, \cdots, X_p = 0, X_{p+1} = 1, \cdots X_{p+q} = 1, X_{p+q+1} = 2, \cdots X_{p+q+r} = 2}
\end{align}
\begin{align}
&= \brak{\frac{P}{N}\times \cdots \times \frac{P-\brak{p-1}}{N-\brak{p-1}}}\brak{\frac{Q}{N-p}\times \cdots \times \frac{Q-\brak{q-1}}{N-\brak{p+q-1}}} \brak{\frac{R}{N-\brak{p+q}} \times \cdots \times \frac{R-\brak{r-1}}{N-\brak{p+q+r-1}}}\\
&= \frac{P! Q! R! \brak{N-\brak{p+q+r}}!}{N!\brak{P-p}!\brak{Q-q}!\brak{R-r}!} \\
&= \frac{P! Q! R! \brak{N-n}!}{N!\brak{P-p}!\brak{Q-q}!\brak{R-r}!} \label{eq:1}
\end{align}
The expression \ref{eq:1} is one possibility of draws such that there are - $p$ red, $q$ blue, $r$ greem marbles. There are $\frac{n!}{p!q!r!}$ such terms, All have the same probability. Hence the required probability is given by, 
\begin{align}
&= \frac{n!}{p!q!r!} \frac{P! Q! R! \brak{N-n}!}{N!\brak{P-p}!\brak{Q-q}!\brak{R-r}!}\\
&= \frac{P!}{p!\brak{P-p}!}\frac{Q!}{q!\brak{Q-q}!}\frac{R!}{R!\brak{R-r}!}\frac{n!\brak{N-n}!}{N!}\\
&= \frac{^{P}C_{p} \times ^{Q}C_{q} \times ^{R}C_{r}}{^{N}C_{n}}
\label{eq:2}
\end{align}
%\begin{align}
%\brak{\frac{p}{N}\times \frac{p-1}{N-1} \cdots \times \frac{1}{N-\brak{p-1}}}\brak{\frac{q}{N-\brak{p-1}-1}\times \frac{q-1}{N-\brak{p-1}-2} \cdots \times \frac{1}{N-\brak{p-1}-\brak{q-1}}} \brak{\frac{r}{N-\brak{p-1}-\brak{q-1}-1}\times \frac{r-1}{N-\brak{p-1}-\brak{q-1}-2} \cdots \times \frac{1}{N-\brak{p-1}-\brak{q-1}-\brak{r-1}}}
%\end{align}
\newpage
In this question, total marbles in the box are 60 - 10 red, 20 blue, 30 green. Out of which 5 balls are drawn. Hence we have,
\begin{align}
N = 60, \, P = 10, \,Q = 20,  \, R = 30, \,n = 5
\end{align}
\begin{enumerate}
\item The probability that all drawn marbles are blue implies
\begin{align}
p = 0,\,  q = 5, \, r = 0
\end{align}
From \eqref{eq:2} we get the probability as,
\begin{align}
&= \frac{^{20}C_{5}}{^{60}C_{5}}
\end{align}
\item The probability that the drawn marble contains atleast 1 green. This event is complement to the event where no marble drawn is green, Its probability is given by,
\begin{align}
&= \frac{^{30}C_{5}}{^{60}C_{5}}
\end{align}
Hence the required probability is given by,
\begin{align}
1 - \frac{^{30}C_{5}}{^{60}C_{5}}
\end{align}
\end{enumerate}
\end{enumerate}
\end{document}